\documentclass{SureshLimkar}
\renewcommand{\baselinestretch}{1.5}

%Enter Suitable Information
%May Differ
%-----------------------------------------------------------------
\def \statustitle {Mr.} %Title (Mr., Ms., Mrs.)
\def \thesistype {A Seminar Report} %Thesis Type
\def \thesistitle {\lq\lq Real-Time Dynamic Traffic Light Timing Adaptation Algorithm and Simulation Software\rq\rq} %Thesis Title
\def \authorname {Mital S. Potey [Roll No: 63]} %Author
\def \nameoncerti {\textbf{Mital S. Potey}} %Author on Certificate
%\def \RegNo {Roll No: 61,63,77,79} %Registration No.
\def \course { T.E.} %Degree
\def \Specialisation {Computer} % only for network student 
\def \guidename {Mrs. Minal Zope} %Guide Name
\def \depthead {Mrs. S.N.Zaware} %Name of HoD
\def \department {Computer Engineering} %Department
\def \year {2012-13} %Academic Year
%-----------------------------------------------------------------

%Common for all. Can be optional
%-----------------------------------------------------------------
\def \logofile {GHRCEMlogo.jpg} 
\def \insti {AISSMS's Institute Of Information Technology}
\def \place {Shivaji Nagar, Pune - 01}
%-----------------------------------------------------------------
\def \acktext %Enter Acknowledgement here
{
\hspace {0.4 in}Apart from my own, the success of this report depends largely on the encouragement and guidelines of many others. I am especially grateful to my guide \textbf{Mrs. Minal Zope} who has provided guidance, expertise and encouragement. \\

\hspace {0.3 in}I express my heartfelt gratefulness to \textbf{Mr. Suresh Limkar} and \textbf{Mrs. S.N.Zaware}, Head of Computer Engineering Department, AISSMS's IOIT, for their stimulating supervision whenever required during my seminar work. I am also thankful to the staff of \textbf{Computer Engineering Department} for their cooperation and support. \\

\hspace {0.3 in}I would like to put forward my heartfelt acknowledgement to all my classmates, friends and all those who have directly or indirectly provided their overwhelming support during my seminar work and the development of this report.
}

\def \abstrtext %Enter Abstract Here
{
\hspace {0.5 in}This document provides details on the implementation of a \textbf{Real-Time Dynamic Traffic Light Timing Adaptation Algorithm and Simulation Software} which helps in reducing congestions at traffic signals. The proposed Dynamic traffic control avoids problems that usually arise with standard traffic control systems. This Dynamic traffic light technique deals with a multi-vehicle, multilane, multi road junction area. It provides an efficient time management scheme, in which a dynamic time schedule is worked out in real time for the passage of each traffic column. The real time operation of the system emulates the judgment of a traffic policeman on duty. The number of vehicles in each column and the routing are proprieties, upon which the calculations and the judgments are based. \\

\textsl{\textbf{Keywords-} Dynamic traffic sequence, dynamic time schedule.}
}

\begin{document}

\maketitlepage %Title page

\newpage

\makecertipage %Certificate

\newpage

\makeackpage %Acknowledgement

\newpage

\makeabstrpage %Abstract

\tableofcontents

\listoffigures

\listoftables

%\newpage

\chapter{Introduction}
\section{Motivation}
\hspace {0.5 in}\textbf{Real-Time Dynamic Traffic Light Timing Adaptation Algorithm and Simulation Software} is a new form of system which is efficient as well as time saving. The operation of standard traffic lights which are currently deployed in many junctions, are based on predetermined timing schemes, which are fixed during the installation, and remain until further resetting. The timing is no more than a default setup to control what may be considered as normal traffic. Although every road junction by necessity requires different traffic light timing setup, many existing systems operate with an over simplified sequence. This has instigated various ideas and scenarios to solve the traffic problem. To design an intelligent and efficient traffic control system, a number of parameters that represent the status of the road conditions must be identified and taken into consideration.

\section{Background}

\hspace {0.5 in}The system explained in this paper was implemented in Sri Lanka. The timing on these circuits are decided on having some prior knowledge about the 
\hspace {0.5 in} Average traffic flow patterns 
\hspace {0.5 in}Average time to pass through the intersection etc.
\hspace {0.5 in}Under this fixed time mode, the signal continuously cycles regardless of actual traffic demand. As a solution to this problem, a system is proposed which count the number of vehicles entering the intersection from each direction and then the traffic lights are operated giving priority to the lane with the highest traffic density. 

\begin{figure}[h]%
\centering
\subfloat{\includegraphics[scale=0.075]{newsys.jpg}}%
\caption{Multilane traffic sequence flow}%
\label{Multilane traffic sequence flow}%
\end{figure}

This is an attempt to use an intelligent system, which replicates having a human controlling traffic, thus can avoid giving the green light to an empty road while motorists on a different route are stopped.

\\
\chapter{VEHICLE DETECTION}
\section{Condition}
\hspace {0.5 in}The main unit of the system is the vehicle detecting sensor which counts the number of vehicles that moves towards the intersection. There are many different ways for vehicle detection. One of the technologies, which are used today, consists of wire loops placed in the pavement at intersections. They are activated by the change of electrical inductance caused by a vehicle passing over or standing over the wire loop.  Other than that, Infra-red object detectors, acoustic sensors etc can be used.  As for the vehicle detector in this project, HMC1052L Magneto resistive Sensor has been used, which is a product from Honeywell HMC series. The sensors task is to detect the presence of vehicles.  
\begin{figure}[h]%
\centering
\subfloat{\includegraphics{chip.jpg}}%
\caption{ HMC1052L }%
\label{HMC1052L}%
\end{figure}
\section{How Sensors work}
\hspace {0.5 in}Due to the fact that almost all road vehicles have significant amounts of ferrous metals in their chassis (iron, steel, nickel, cobalt, etc.), magnetic sensors are a good choice for detecting vehicles. The nature offers the earth’s magnetic field that permeates everything between the south and north magnetic poles. So when a vehicle, that is a metal body, moves through the field, lines of flux bends. As the lines of magnetic flux group together (concentrate) or spread out (de-concentrate), a magnetic sensor placed nearby will be under the same magnetic influence the vehicle creates to the earth’s field. However because the sensor is not intimate to the surface or interior of the vehicle, it does not get the same fidelity of concentration or de-concentration and with increasing standoff distance from the vehicle, the amount of flux density change with vehicle presence drops of at an exponential rate. This has a very big advantage in vehicle detection as there is no false detection of a vehicle in an adjacent lane or vehicle in the adjacent parking spot.
\begin{figure}[h]%
\centering
\subfloat{\includegraphics{chip1.jpg}}%
\caption{}%
\caption{ HMC1052L }%
\label{HMC1052L}%

\end{figure}


\chapter{Dynamic Traffic Sequence Algorithm}
\section {Intelligent selection of Traffic}
\hspace {0.5 in} The pattern of the change in earth magnetic field when a vehicle is moving is different for different vehicles, depending on the amount of metal the vehicle contains and the speed of the vehicle.  Fig. 01, illustrates the quantized value of the of the output voltage of the sensor, as a bus was moving slowly. There is a sudden change in the magnetic field as the vehicle moves.  By generalizing all the patterns obtained for different types of vehicles in different conditions, an algorithm has been developed to detect vehicles.  
\begin{figure}[h]%
\centering
\subfloat{\includegraphics[scale=0.675]{chart.jpg}}%
\caption{Traffic Light State Diagram}%
\label{Traffic Light State Diagram}%
\end{figure}

The algorithm has to be optimized such that one vehicle is counted only once and several vehicles moving together are detected separately. The Fig.  02, illustrates the change in the magnetic field as several vehicles move together. By analyzing many such patterns for different vehicle types,  an algorithm was developed to detect all types of vehicles including heavy vehicles, light vehicles, dual purpose vehicles, three wheelers and motor bicycles. This algorithm was implemented on to a controller and was tested on road.    
The sensing units at each lane transmits the vehicle count of that lane to the central controlling unit periodically, which then calculates the optimum timing for the traffic lights to operate. 
\begin{figure}[h]%
\centering
\subfloat{\includegraphics[scale=0.675]{chart1.jpg}}%
\caption{Traffic Light State Diagram}%
\label{Traffic Light State Diagram}%
\end{figure}

\\

\chapter{IV. TRAFFIC LIGHT TIMING ALGORITHM }


\hspace {0.5 in}The aim in designing and developing this intelligent traffic signal controller is to reduce the waiting time of each lane of the cars and also to maximize the total number of cars that can cross an intersection given the mathematical function to calculate the waiting time. There are several phases of operation in standard traffic models.  For example, 2-phase systems, 3-phase systems etc. The Fig. 3 illustrates the two phases in the 2-phase operation.  As it can be seen the 2- phase systems are only suitable for junctions with less percentage of right turns. \\
\begin{figure}[h]%
\centering
\subfloat{\includegraphics[scale=0.675]{flow.jpg}}%	
\end{figure}
\\
\hspace{0.5 in}This design was only limited to 2-phase systems, since 90 of the four way intersections in Sri Lanka are operated using this model. However it can easily be upgraded for 3-phase and 4- phase systems too.\\
\hspace Algorithm:- Step 01: Through a field survey find the following parameters
lane width: width of the lane at the intersection\\
\hspace Pedestrian flow for hour: Usually at an intersection there is a pedestrian flow. This should be set as low (0-99), moderate (100-599), High(600-1199) , Very High(more than 1200) 
\hspace Percentage of right-turns, left-turns, straight-through in each lane. \\
\hspace Step 02: Calculating the Passenger Car Volume (PCV)\\
\hspace PCV = (Percentage of Heavy vehicles * 1.75) + (Percentage of buses*2.25)+(Percentage of light commercial vehicles*1.00)+(Percentage of mini-buses*1.25)+(Percentage of motor cycles*0.50)+(Percentage of cars*  1.00)\\
\newpage\\
\hspace Step 03: Define the all-red time, red-amber time.\\ 
\hspace All-red time is the time that all the red lights at the intersection are illuminating at the beginning of a timing cycle. Red-amber time is the time that both the red and the amber lights are illuminating.\\
\hspace Step 04: Calculate the Critical Lane Volume (CLV) CLV = Maximum value between sum of movements disturbing each other.This should be calculated for both phase 01 and 02.\\
\hspace Step 05: Calculate the cycle time using the standard ‘Webster Formula’\\ 
 Cycle time C0,\\
\hspace C0 = ( 1.5L+5 ) / [1-( ΣCLVi / S ) ]  
S=1800 pcu/hour (for a 2-phase system)\\
\hspace L=lost time due to acceleration + all red time + red-amber time\\
\hspace Step 06: Calculate the effective green time The time for the green light to illuminate at each phase.\\
\hspace gi = [(C0- L) CLi ]/ ΣCLVi\\ 
\hspace Where i=1,2 (i.e.,must be calculated for both phases) \\
\hspace Step 07: Calculate the actual green time 
Actual green time is the effective green time minus the losses due to acceleration, all-red time, red-amber time. Gi= gi- ai+ li\\
\hspace Where i=1,2 li =loss time ai =amber time\\ 
\hspace Step 08: Calculate actual the red time\\
Ri = C0- (Gi + ai + (ai / li )CLi         
\\

\chapter{V. TRAFFIC MODELLING SOFTWARE}

\hspace{0.5 in}A traffic modeling software was developed in-order to calculate the timing for the traffic lights to operate, using the standard ‘Webster’ formula and then to simulate the operation of the traffic lights in a Graphical User Interface.  In order for the traffic modeling software to operate real time the central controller which receives the vehicle count was interfaced to the PC using the serial connector.  The Fig. 4, illustrates the interface to which all the data collected during the field survey (explained above) should be entered.

\begin{figure}%
\centering
\subfloat{\includegraphics[scale=0.5]{simu.jpg}}%
\caption{Initial State}%
\label{Traffic Light State Diagram}%
\end{figure}



\hspace{0.5 in}The pedestrian flow for hour can be selected from the entries in the drop down list as Low, Moderate, High, etc.  The traffic arrival percentage for each lane can be entered by selecting the lane from the drop down list as North, South, East, West.  PCV can be calculated using the form that appears when the Calc button in the signal timing analyzer is clicked.  After all the field data is entered the OK button is clicked which opens the traffic signal simulator.

The traffic signal simulator illustrates the change of traffic signal timing as it receives data from the central controller connected via the serial port.  Operation of the traffic signal simulator can be described in the following steps:\\ 
\hspace Step 01: ‘Read’ button. Starts reading the serial port of the computer. All the controlling signals passed between the sensors and the central controller are displayed. When the vehicle count is received a message box is displayed prompting the user to calculate the traffic timing for the new values.\\
\hspace Step 02: Calculate’ button Calculates the optimum cycle time, green time and red time for the lights to operate. The resultant values are displayed.\\
\hspace Step 03: ‘Simulate’ button Simulates the operation of traffic lights according to the new timing calculation.
\begin{figure}[h]%
\centering
\subfloat{\includegraphics[scale=0.675]{emu.jpg}}%
\label{Emulator}%
\end{figure}
\\
\chapter{V. RESULT}
\hspace{0.5 in}Results obtained for the vehicle detection algorithm: The vehicle detection algorithm developed is implemented in the microcontroller and it was tested on the road.  The peaks shown in the Fig.  6 were detected as vehicles moving nearby by the microcontroller correctly. Here, false vehicle detection can occur due to large magnetic field variation when a huge vehicle moves in the adjacent lane. But selecting proper threshold value, this effect can be reduced in the algorithm.
Although natural changes in the earth magnetic field can also cause false, it was eliminated by using the generalized pattern for vehicle detection.

\begin{figure}[h]%
\centering
\subfloat{\includegraphics[scale=0.675]{res.jpg}}%
\label{Result Graph}%
\end{figure}
\\

\chapter{V. CONCLUSION}
\hspace{0.5 in}The prototype of the intelligent traffic light system has 
successfully been designed and implemented. 
Increasing the number of sensors to detect the presence of 
vehicles can further enhance the accuracy of the traffic light 
system. As to increase the reliability of the system, it is 
expected to introduce an automatic switch which changes to 
fixed time mode, during a failure of the dynamic traffic light 
system. Further, the traffic lights may turn off late at night 
when traffic is very light. 
 Another room of improvement is to have magneto resistive 
sensors replaced with an imaging system/camera system so 
that it has a wide range of detection capabilities, which can be 
enhanced and ventured into a perfect traffic system.

\begin{thebibliography}{9}
\bibitem{label0}Rogger P. Roess, Elena S Prassas ,and William R. McShane,” Traffic Engineering”.
\bibitem{label1}  Cotter W. Soyre,” Complete Wireless Design”.
\bibitem{label2} Solomon, S., 1999. Sensors Handbook. McGraw- Hill, New York.
\bibitem{label3} Honeywell. Vehicle Detection Using AMR Sensors.

\bibitem{label4} Anna Hac: "Wireless sensor network designs". John Wiley & Sons Ltd, 2003
\bibitem{label5} "A Realtime Dynamic Traffic Control System Based on Wireless Sensor Network"    CHEN Wenjie, CHEN Lifeng, CHEN Zhanglong, TU Shiliang ,   Department of Computer Science and Engineering, Fudan University system.
\bibitem{label6}www.wikipedia.com

\bibitem{label7}www.youtube.com

\bibitem{label8}thenounproject.com

\end{thebibliography}

\end{document}

