\documentclass{SureshLimkar}
\renewcommand{\baselinestretch}{1.5}

%Enter Suitable Information
%May Differ
%-----------------------------------------------------------------
\def \statustitle {Mr.} %Title (Mr., Ms., Mrs.)
\def \thesistype {A Seminar Report} %Thesis Type
\def \thesistitle {\lq\lq Dynamic traffic light controller using machine
vision and optimization algorithms\rq\rq} %Thesis Title
\def \authorname {Dhruv G. Sangvikar [Roll No: 67]} %Author
\def \nameoncerti {\textbf{Dhruv G. Sangvikar}} %Author on Certificate
%\def \RegNo {Roll No: 61,63,77,79} %Registration No.
\def \course { T.E.} %Degree
\def \Specialisation {Computer} % only for network student 
\def \guidename {Mrs. Minal Patil} %Guide Name
\def \depthead {Mrs. S.N.Zaware} %Name of HoD
\def \department {Computer Engineering} %Department
\def \year {2012-13} %Academic Year
%-----------------------------------------------------------------

%Common for all. Can be optional
%-----------------------------------------------------------------
\def \logofile {GHRCEMlogo.jpg} 
\def \insti {AISSMS's Institute Of Information Technology}
\def \place {Shivaji Nagar, Pune - 01}
%-----------------------------------------------------------------
\def \acktext %Enter Acknowledgement here
{
\hspace {0.4 in}Apart from my own, the success of this report depends largely on the encouragement and guidelines of many others. I am especially grateful to my guide \textbf{Mrs. Minal Zope} who has provided guidance, expertise and encouragement. \\

\hspace {0.3 in}I express my heartfelt gratefulness to \textbf{Mr. Suresh Limkar} and \textbf{Mrs. S.N.Zaware}, Head of Computer Engineering Department, AISSMS's IOIT, for their stimulating supervision whenever required during my seminar work. I am also thankful to the staff of \textbf{Computer Engineering Department} for their cooperation and support. \\

\hspace {0.3 in}I would like to put forward my heartfelt acknowledgement to all my classmates, friends and all those who have directly or indirectly provided their overwhelming support during my seminar work and the development of this report.
}

\def \abstrtext %Enter Abstract Here
{
\hspace {0.5 in}This paper presents a fuzzy traffic controller that in
an autonomous, centralized and optimal way, manages traffic flow
in a group of intersections. The system obtains information from
a network of cameras and through machine vision algorithms can
detect the number of vehicles in each of the roads. Using this
information, the fuzzy system selects the sequence of phases that
optimize traffic flow globally. To evaluate the performance of
the controller, a scenario was developed where it was possible
to simulate through artificially created videos two adjacent
intersections. System performance was compared versus fixed time controllers
An improvement of 20% in performance was observed.\\

\textsl{\textbf}
}

\begin{document}

\maketitlepage %Title page

\newpage

\makecertipage %Certificate

\newpage

\makeackpage %Acknowledgement

\newpage

\makeabstrpage %Abstract

\tableofcontents

\listoffigures

\listoftables

%\newpage

\chapter{Introduction}
\section{Motivation}
\hspace {0.5 in}\textbf{Dynamic traffic light sequence} is a new form of system which is efficient as well as time saving. The operation of standard traffic lights which are currently deployed in many junctions, are based on predetermined timing schemes, which are fixed during the installation, and remain until further resetting. This leadts to a very ineffiecient system. The
timing is no more than a default setup to control what may be considered as normal traffic. Although every road junction by necessity requires different traffic light timing setup, many existing systems operate with an over simplified sequence. This has instigated various ideas and scenarios to solve the traffic problem. To design an intelligent and efficient traffic control system, a number of parameters that represent the status of the road conditions must be identified and taken into consideration.This paper tries to analyze the problem using machine vision and other optimization algorithms.

\section{Background}

\hspace {0.5 in}Nowadays a serious mobility
problem, which affects a great part of the citizens and
harms drastically its productivity and competitiveness [4]. One
of the main reasons, which contributes to this situation, is
the use of inefficient and obsolete traffic controllers, which
are not capable to manage in an efficient way the traffic flow
in the roads of the city. These fixed time controllers, require
a periodical configuration based on statistical flow analyses,
which generally do not reflect in an accurate way the real
traffic flow conditions [1].
In order to solve this problem, new control techniques
have been developed, allowing the creation of completely
autonomous systems, that based in the data collected by a
set of sensors (inductive, capacitive, acoustical), are able to
manage in an optimal and dynamical way the vehicular flow.

\hspace {0.3 in}To avoid this problem, in this work a completely autonomous dynamical controller was developed, which is capable of manage in a coordinated and centralized way, the
state of the traffic lights in a simulated scenario using the
information provided by a set of cameras. This kind of
sensor gives the system great installation flexibility, due to
the possibility of strategic location within the control zone,
avoiding the problems described above and increasing the
durability, efficiency and profitability of the system.\\

\hspace{0.3 in}The main contribution within the development of this controller is the use of a vehicular detection algorithm, which
allows it to identify in an accurate way the number of vehicles
present in each road. Besides, the controller has a diffuse
optimization algorithm, which using the data provided by the
detection algorithm switches the state of the traffic lights,
ensuring a continuous, homogeneous, and fair traffic flow.

\section{Need}

\hspace {0.5 in}The objective of the paper is study, how to preempt the normal working of a traffic light signal. This preemption can not only save time, but also lead to a better traffic management. This can further be helpful in reduction of road accidents, thus reduction in loss of life and property. Dynamic signals can also facilitate in the smooth and easy movement of the emergncy vehicles like ambulance, fire truck, police cars, etc.

\chapter{BACKGROUND AND PREVIOUS RESEARCH}
\section{Traffic controllers}
\hspace {0.5 in}There are two main kinds of traffic controllers: static ones
and dynamical ones. The first ones are those where a sequence
of actions previously programed are followed, while the other
kind makes use of a certain acquisition method, which allows
the system to identify the state of the traffic flow on the roads
and guide his actions to optimize the traffic flow. 

\begin{figure}[h]%
\centering
\includegraphics{trafficcontroller.jpg}%
\caption{Traffic controller system block diagram}%
\label{Traffic Controller system}%
\end{figure}


\hspace {0.5 in}A phase is a traffic signal
which allows a flow of non-conflictive movements. For example, Phase 1 showed in Figure 1 allows traffic flow from west
to east and vice versa. In the same way, a succession of phases
which is repeated continuously is considered a cycle. Figure
1 shows a cycle made-up of 4 phases. Finally, coordination is
the action of programing the signalized intersections in such a
way, that the flow of a corridor can achieve a constant speed
without detentions, generating what is known as green waves.

\begin{figure}[h]%
\centering
 \includegraphics{phase.jpg}%
\caption{Phase}%
\label{Phases in traffic system}%
\end{figure}

\section{Object detection}
%\begin
\hspace{0.5 in}In the object detection field, there are two main strategies
concerning the vehicle detection task: the first one is based
on background and optical flow estimation, while the second
one uses machine learning techniques. Background estimation
analyzes the difference between a predefined model (image)
of an empty road and an image of the incoming traffic,
obtaining perturbations, that overlapped to the predefined
model are interpreted as vehicles [16], [17]. A great portion
of the investigations about machine learning methods has been
framed to the ‘on-road’ vehicle detection (a camera installed
inside a car), instead of applications for traffic control on
intersections. Examples of methods used within this area are:
Boosted Cascade of haar Features, Sift (Scale Invariant Feature
Transform) matching and neural networks.
In the same way, there is certain terminology which is
important and will help understand this portion of the work.
Classifier is an operator which uses the features of a data
set, identifying the class or group to which each of these
data belongs. Boosting is a meta-algorithm, which pretends
to create a strong classifier through the addition of weak
classifiers, and a feature is considered as an important piece
of information [9].
There are plenty of investigations in the area of vehicle
detection through images; the following are some of the most
important researches in this field: in [10] and [11], an on
road vehicle detector was developed through a Haar like
feature detector, obtaining an accuracy detection of 88,6%
and 76% respectively. In [12] and [13] authors used the
background estimation technique with an efficiency rate over
90% in both cases. On the other hand, in [14] a morphological
edge detector (SMED) was developed, which presents more
insensitiveness to illumination changes than the background
estimation, obtaining an accuracy of 95%.

\chapter{Traffic Control System}
\hspace {0.5 in}This computer is connected
to a centralized server that processes the information and
executes the detection and control algorithms. Finally all the
decisions taken are sent back to each computer, which change
the traffic lights depending on these orders.
\newpage
\section{Fuzzy control system}
The controller developed is based on the model presented by
Lee et al. in [3], which evaluates not only the variables related
to the controlled intersection, but also analyzes the variables
related to traffic flow at nearby intersections. This allows
the system to operate in a coordinated way, thus generating
so-called "green waves", avoiding unnecessary detentions for
vehicles traveling through the roads and avoiding sending
vehicles to areas of high congestion. The following are its main components:
\\

\begin{enumerate}
	\item {Next phase module}
	\item {Observation module}
	\item {Decision module}
	
\end{enumerate}

\subsection{Next Phase Module}
This is responsible for selecting among all the phases
that are not active, the one whose level of urgency is
higher. To achieve this, this module evaluates the urgency
of each of the flows associated with each phase and the
average of these values will be the level of urgency of
the phase analyzed. For example, the level of urgency
of the phase showed in Figure 4 is the average of the
values obtained evaluating the north-south flow (green)
and north-east flow (red).

\begin{figure}[h]%
\centering
 \includegraphics[scale=0.675]{nextphase.jpg}%
\caption{Next phase determination}%
\label{Next phase}%
\end{figure}

\hspace {0.5 in} To obtain the level of urgency of each flow, four variables
are evaluated: NumCar is the number of vehicles waiting
for the green signal, in Figure 4 they are represented in
color blue; RedTime represents the number of periods that
the evaluated phase has been deactivated; NumCarAnt is
an estimate of the number of vehicles that could arrive
from the lanes before the intersection, and FNumCar is
the number of vehicles on the road in front of the intersection, for the north-south flow in Figure 4 this variable
is represented with green color. In this way, the variables
RedTime and NumCar reflect traffic conditions locally,
while NumCarAnt and FNumCar allow the system to
coordinate different neighboring intersections.
Figure 5 shows the Fuzzy Set of this module and Table I
presents some of its rules. For example, R2 states that if
the number of vehicles waiting to cross is High (NumCar
= H), the number of periods in which the analyzed phase
has not been active is High (RedTime = H) and the
number of vehicles waiting in the following lane is Low
(FNumCar = L), then the urgency of this phase will be very high.

\begin{figure}[h]%
\centering
 \includegraphics[scale=0.675]{nextrule.jpg}%
\caption{Next phase rules}%
\label{Next phase rules}%
\end{figure}


\subsection{Next Phase Module}
\hspace{0.5 in} This module is responsible for assessing traffic conditions
for the active phase and determines, how timely it would
be to stop that phase. The fuzzy rules of this module
have two inputs and one output: ONumCar indicates
the number of cars that still are on stanby; OFNumCar
represents the number of vehicles at the next intersection
and Stop is the output of the module and indicates,
whether or not should be necessary to stop the phase.
The behavior of the input variables is very similar to
variables NumCar and FNumCar, therefore their fuzzy
sets are equal. Figure 6 shows the Fuzzy Set for the Stop
variable.
Table II presents some rules of this module. R4 indicates
that if the number of vehicles waiting for the active phase
is still high (ONumCar = H) and the number of vehicles
in the following lane is high too (OFNumCar = H), then
the phase must be stopped (Stop = Yes). This is because
it would be a waste of time to allow a flow that will be
obstructed later.

\begin{figure}[h]%
\centering
 \includegraphics[scale=0.675]{obs.jpg}%
\caption{Some rules of observation module}%
\label{some rules of observation module}%
\end{figure}

\subsection{Decision Module}

\hspace {0.5 in} This module decides whether or not change the active
phase. The inputs in this module are Urgency and Stop
and the output is Decision. The two input variables are the
outputs of the modules ’Next Phase’ and ’Observation’,
respectively. The module changes the active phase for
that which is candidate, as long as the result of the
defuzzification is above a given threshold.
Table III shows some of the rules of this module. The
first rule indicates that, although the candidate phase has
a medium congestion (Urgency = M), if the Stop level of
the active phase is low (Stop = N), then the module will have to keep the same phase (Decision = N, no change).
The Fuzzy Set of this module is presented in Figure
7 (Urgency and Stop variables appear in the previous
modules).\\

\begin{figure}[h]%
\centering
 \includegraphics[scale=0.675]{dec.jpg}%
\caption{Decision by decision module}%
\label{Decision by decision module}%
\end{figure}

\section{Detection Algorithm}

\hspace {0.5 in} The implementation of this
method is made up of two big phases, one dedicated to the
training of the classifiers through a machine learning algorithm
called Adaboost and the construction of the cascade, and the
other where the detection is adapted to the own needs of the
interest object and the context where these objects exist.\\

\hspace {0.5 in} For the present work, the positive examples set is extracted
 from traffic videos of several points of the city. From these videos 6364 images are obtained, for
each one of these images true regions are annotated; 10050
true regions were found, thus obtaining the same number of
positive examples. In order to obtain the negative example
set, videos from daily scenes of parks and walkways are used,
besides the image datasets from Google, CALTECH, CMU,
TU Darmstadt, UIUC, VOC2005 y TU GRAZ are used too,
from these, 8131 images are extracted in which do not exist
a single car with frontal view.
\\
\hspace {0.5 in} The performance of the whole object detection system
depends on several training parameters of the strong and weak
classifiers, as well as the cascade itself, some examples of
these parameters are: the size of the example sets, number of
stages of the cascade, type of weak classifier... etc. In order
to estimate the optimal values for these parameters, a series
of experiments based on the work of were conducted, but
for the specific case of vehicles as interest objects.
\\

\begin{figure}[h]%
\centering
 \includegraphics[scale=0.675]{detection.jpg}%
\caption{Detection system and comparison}%
\label{Detection and comparison}%
\end{figure}

\chapter{EXPERIMENTAL FRAMEWORK}

\hspace {0.5 in} In order to verify system performance in a controlled, but
projectable environment, it was necessary to implement a
test scenario using artificial videos. For this, an algorithm
was developed using MATLAB, which is capable of creating
random videos that simulate traffic flow in a lane
\\

\begin{figure}[h]%
\centering
 \includegraphics[scale=0.675]{framework.jpg}%
\caption{Multilane traffic sequence flow}%
\label{Multilane traffic sequence flow}%
\end{figure}

\hspace {0.5 in} In order to compare the performance of the developed
system over fixed-time controllers, both of them were tested
under the same traffic conditions.
\\


\chapter{RESULTS AND CONCLUSIONS}

\hspace {0.5 in} For each controller (Fixed-time and Fuzzy), each of the
plans was executed for 20 minutes. In order to compare the
performance of each of them, two control variables were
evaluated: the first one was the average delay time of each of
the simulated vehicles, and the other was the number of cars
that each controller was able to handle in the same period of
time.\\

\hspace {0.5 in} the results show that the developed
system reduces the time delay caused by unnecessary stops in about 20%. It is also important to note that the system
was able to adapt quickly and efficiently in those plans where
there was a change in the level of congestion (7, 8, 9 and 10),
outperforming the standard controller up to 26. 
\\

\hspace {0.5 in} The controller developed was able to deal with a number of vehicles much
larger than the standard controller, improving performance by
up to 28.45%.
\\

\hspace {0.5 in} Besides both cascades presented similar performance in
terms of processing speed, reaching a detection rate between
22 and 27 frames per second on images of 320x240 pixels. This advantage puts the chosen method above
others, like background estimation and optic flow estimation.
\\

\hspace {0.5 in} Unlike vehicle detection methods based on optic flow calculation, the constructed detector is able to locate the vehicles
even when these are stopped. In the same way, unlike methods
based on tripline techniques, the constructed method does not
show problems if vehicles change lanes intermittently or if
these does not transit through certain predefined areas of the
image.
\\

\hspace {0.5 in} On the other hand, the results show that the proposed controller’s performance far exceeds that of fixed-time controllers,
and also this can be optimally adapted to a large number of
situations. Similarly, the project is applicable to any situation,
regardless of the number of intersections or distribution.
\\

\hspace {0.5 in}Finally, it is observed that machine vision algorithm proposed for the detection of vehicles, presents a clear disadvantage, as is the lack of robustness to the presence of occlusions
of the objects of interest, requiring that these occlusions are
less than the 10% of the total area of the object. Therefore
the location and height at which video sensors are installed
should be considered, so that the level of occlusion between
vehicles could be reduced.
\\

\begin{figure}[h]%
\centering
 \includegraphics[scale=0.675]{results.jpg}%
\caption{Results and advantages at a glance}%
\label{Results}%
\end{figure}


